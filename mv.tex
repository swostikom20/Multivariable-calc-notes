\documentclass{article}
\usepackage[utf8]{inputenc}
\usepackage{a4wide}
\usepackage{dsfont}
\usepackage{amsmath}
\usepackage{upgreek}
\usepackage{ mathrsfs}
\usepackage{setspace} \doublespacing
\title{MV notes}
\author{Om Swostik Mishra}
\date{}
\begin{document}
\maketitle

\begin{flushleft}
\textbf{Cantor set($\mathscr{C}$)}:     

$C_0=[0,1]$

$C_1=[0,1]-(\frac{1}{3},\frac{2}{3})$ and so on (Basically, keep removing the middle one third)

$\mathscr{C}=\overset{\infty}{\underset{i=0}{\bigcap}} \mathcal{C}_i$

(I) $\mathscr{C}\neq \phi$

(II) $\mathscr{C}$ is closed (hence, compact)

(III) $\mathscr{C}$ has length 0

(IV) $\mathscr{C}$ is totally disconnected

\textbf{Exercise}: Can you characterize the elements of $\mathscr{C}$?

\textbf{Integration}:

\underline{\textbf{In one variable}}:

\textbf{Definition}: A partition $P$ of $[a,b]\subset\mathds{R}$ is a collection $t_0=a\leq t_1\dots \leq t_k=b$

Rectangle in $\mathds{R}^n$: $[a_1,b_1]\times [a_2,b_2]\dots \times[a_n,b_n]=A$

If $P_i$ is a partition of $[a_i,b_i]$, $P=(P_1,P_2,\dots P_n)$ is a partition of $A$.

$f:A\rightarrow \mathds{R}$ is a bounded function. ($P$ is a partition of $A$)

If $S=$ rectangle in $P$, $m_S(f)=\inf_{x\in S} f(x)$, $M_S(f)=\sup_{x\in S} f(x)$

$L(f,P)=\sum_{S\in P}^{} m_S(f)\nu(S)$

$U(f,P)=\sum_{S\in P}^{} M_S(f)\nu(S)$  ($\nu(S)=$ volume of $S$)

$L(f,P)\leq U(f,P)$ 

If $P$ and $P{'}$ are partitions, $P^{'}$ is said to be an refinement of $P$ if any subrectangle $S^{'}$ in $P^{'}$ is contained in contained in some subrectangle $S$ in $P$.

If $P=(P_1,P_2\dots P_n)$ and $Q=(Q_1,Q_2\dots Q_n)$, then

$\mathcal{T}=(P_1\cup Q_1,\dots ,P_n\cup Q_n)$ is a refinement of both $P$ and $Q$.

\textbf{Proposition}: If $P^{'}$ is a refinement of $P$, then $L(f,P)\leq L(f,P^{'})\leq U(f,P^{'})\leq U(f,P)$.

If $S^{'}\in P^{'}$, then $S^{'}\subseteq S$ for some $S\in P$, $m_{S^{'}}(f)\geq m_S(f)$, $M_{S^{'}}\leq M_S(f)$

$S=\overset{m}{\underset{i=1}{\bigcup}} S_i^{'}$, $S_i^{'}\in P^{'}$

$\sum_{i=1}^{m} m_{S^{'}}(f).\nu(S^{'})\geq \sum_{i=1}^{m} m_S(f).\nu(S^{'})=m_S(f)\sum_{i=1}^{m} \nu(S_i^{'})=m_S(f) \nu(S)$

Summing up, $L(f,P^{'})\geq L(f,P)$

Similarly, $U(f,P^{'})\leq U(f,P)$

\textbf{Corollary}: If $P$ and $P^{'}$ are any two partitions: $L(f,P^{'})\leq U(f,P)$

Take $Q=$ common refinement of $P$, $P^{'}$.

Then, $L(f,P^{'})\leq L(f,Q) \leq U(f,Q) \leq U(f,P)$ 

$\sup_{P} L(f,P)\leq \inf_{P^{'}} U(f,P^{'})$

$f: A\rightarrow \mathds{R}$: a bounded function is said to be integrable if $\sup_{P} L(f,P)=\inf_{P} U(f,P)$

Value is denoted as $\int_{A}^{} f$

\textbf{Propositon}: $f$ is integrable $\Leftrightarrow$ $\forall \epsilon >0$,
$\exists$ a partition $P$ such that, $U(f,P)-L(f,P)<\epsilon$ 

\textbf{Theorem}: Let $f:A \rightarrow \mathds{R}$ be a bounded function on a closed rectangle $A$. Then $f$ is integrable on $A$ iff the set $B= \{x: f \: \text{is discontinuous at}\: x\}$ has measure 0.

\textbf{Definition}: A set $S\subseteq \mathds{R}^n$ is said to have measure 0 if given $\epsilon>0$ there exists a sequence of closed rectangles $Q_1,Q_2\dots Q_n$, such that $S\subseteq \overset{\infty}{\underset{i=0}{\bigcup}} Q_i$ and $\sum_{i=1}^{\infty} \nu(Q_i)<\epsilon$

\textbf{Exercise}: 

(I) If $\{A_i\}_1^{\infty}$ is a countable collection of measure zero sets,then $\overset{\infty}{\underset{i=0}{\bigcup}} A_i$ has measure zero.

(II)$\mathscr{C}$ has measure zero

\textbf{Definition}: A set $A\subseteq \mathds{R}^n$ is said to have content 0, if given $\epsilon>0$, there exists a finite collection of closed rectangles $Q_1,Q_2 \dots Q_n$ such that $\sum_{i=1}^{n} \nu(Q_i) <\epsilon$

content 0 $\Rightarrow$ measure 0 

\textbf{Theorem}: measure 0 $\Rightarrow$ content 0 if $A$ is compact 

\textbf{Proposition}: If $a<b$, then $S=[a,b]$ doesn't have measure zero.  

Proof(Sketch): Let $S\subseteq \overset{\infty}{\underset{i=0}{\bigcup}} Q_i$

$\sum_{}^{} \nu(Q_i) \geq (b-a)$

\end{flushleft}
\end{document}    